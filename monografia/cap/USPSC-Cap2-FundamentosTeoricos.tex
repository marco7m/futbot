\chapter{Fundamentos Teóricos}

\section{\textcolor{red}{futebol de robos vss}}
\textcolor{red}{Explicar o que é e as regras do futebol de robôs \cite{IEEE2008}}

\section{\textcolor{red}{Conceitos Básicos de Programação Orientada a Objetos}}

\section{Algoritmos Evolutivos}
O algoritmo evolutivo (AE) é uma técnica de otimização importante que vem crescido na ultima década. Devido a sua natureza flexível e comportamento robusto herdado da computação evolutiva, ele se torna um método eficiente para solução de problemas de otimização global e é utilizado principalmente quando se tem uma capacidade de computação limitada e informação insuficiente ou imperfeita. 

Existem diversas subáreas dentro dos AEs, dentre elas estão a programação evolutiva, estratégia evolutiva, programação genética e o algoritmo genético (AG), que é a mais popular dentre elas e foi inspirado nos mecanismos da evolução e genética natural \cite{Srinivas1994}. Todas as subáreas dos AEs trabalham com o princípio comum da evolução simulada de indivíduos utilizando o processo de seleção, mutação e reprodução. Pode-se diferenciar as diferentes subáreas dos algoritmos evolutivos com base na sua implementação e no modo com que elas são aplicadas a um problema em particular \cite{Vikhar2017}.

Na implementação de um algoritmo genético existem algumas fases que devem ser elaboradas para a iteração do algoritmo, elas serão descritas em detalhes a seguir.

\subsection{População Inicial}

O processo de se aplicar um algoritmo genético para a resolução de um problema, se inicia com a criação de uma população inicial. A população inicial é composta por indivíduos, sendo cada um deles, uma solução para o problema gerado geralmente aleatoriamente.

Os indivíduos são representados por um conjunto de parâmetros, sendo cada um desses parâmetros chamado de gene. Os parâmetros do indivíduo são representados como uma sequencia de números. 

Uma outra nomenclatura utilizada para o conjunto dos genes do indivíduo é cromossomo.

\subsection{Função de Aptidão}

A função de aptidão desempenha um papel importante no sucesso do algoritmo genético em encontrar a melhor solução para um problema, pois é ela que classifica o cromossomo com relação à sua performance. Isso é uma ligação importante entre o AG e o sistema.
\cite{Man1996}

\subsection{Reprodução}
Durante a fase de reprodução do algoritmo genético, indivíduos são selecionados aleatoriamente, utilizando uma estratégia que favorite os melhores classificados pela função de aptidão, para gerar descendentes que irão compor a próxima geração.

Após a escolha dos indivíduos, eles passarão por dois processos. O primeiro é o cruzamento, que a partir do cromossomo de dois ou mais indivíduos é gerado o cromossomo de um terceiro indivíduo. Esse novo cromossomo é gerado a partir da combinação dos genes dos indivíduos anteriormente selecionados. Esse processo geralmente não é utilizado em todos os indivíduos, alguns costumam ser escolhidos para passarem seus genes para a próxima geração sem a interferência do cruzamento.

O segundo processo é chamado de mutação, ele é aplicado a cada cromossomo após o processo de cruzamento. Esse processo altera aleatoriamente cada gene em um valor bem pequeno. A mutação é uma etapa muito importante para a busca da melhor solução em todo o espaço de busca, pois ela proporciona uma pequena aleatoriedade na busca, o que ajuda a garantir que nenhum ponto no espaço de busca tem probabilidade zero de ser examinado.
\cite{Beasley1993}

\subsection{Predação}
O processo de predação consiste na remoção de determinados indivíduos, não passando eles para a próxima geração e nem permitindo que eles se reproduzam. Geralmente se escolhe os piores indivíduos de acordo com a função de aptidão para serem removidos ou terem maiores chances de serem removidos, a fim de se diminuir as chances de uma característica indesejada seja passada para a próxima geração, porém pode-se também optar por remover um indivíduo aleatório desde que esse não seja o melhor indivíduo da geração.

\section{\textcolor{red}{Rede Neural Artificial}}

\textcolor{red}{Não feito}

\section{\textcolor{red}{Algoritmo Neuro-evolutivo}}
