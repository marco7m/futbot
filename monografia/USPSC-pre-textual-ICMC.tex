%% USPSC-pre-textual-ICMC.tex
%% Camandos para definição do tipo de documento (tese ou dissertação), área de concentração, opção, preâmbulo, titulação 
%% referentes ao Programa de Pós-Graduação o IFSC
\instituicao{Instituto de Ci\^encias Matem\'aticas e de Computa\c{c}\~ao, Universidade de S\~ao Paulo}
\unidade{INSTITUTO DE CI\^ENCIAS MATEM\'ATICAS E DE COMPUTA\c{C}\~AO}
\unidademin{Instituto de Ci\^encias Matem\'aticas e de Computa\c{c}\~ao}
\universidademin{Universidade de S\~ao Paulo}
\notafolharosto{Vers\~ao original}
%Para versão original em inglês , comente do comando/declaração acima (inclua % antes do comando acima) 
%                       e tire a % do comando/declaração abaixo no idioma do texto
%\notafolharosto{Original version} 
%Para versão revisada, comente do comando/declaração acima (inclua % antes do comando acima) 
%                       e tire a % do comando/declaração de um dos comandos abaixo em conformidade com o idioma do texto
%\notafolharosto{Vers\~ao revisada}
%\notafolharosto{Final version}

% ---
% dados complementares para CAPA e FOLHA DE ROSTO
% ---
\universidade{UNIVERSIDADE DE S\~AO PAULO}
\titulo{Modelo para teses e disserta\c{c}\~oes em \LaTeX\ utilizando a classe USPSC para o ICMC} 
\titleabstract{Model for theses and dissertations in \LaTeX\ using the USPSC class to the ICMC}
% para a versão em inglês, utilize os comandos abaixo
%idioma{eng}
%\titulo{Model for theses and dissertations in LaTeX using the USPSC class to the ICMC}
%\titleabstract{Modelo para teses e disserta\c{c}\~oes em LaTeX utilizando a classe USPSC para o ICMC}

\autor{Jos\'e da Silva}
\autorficha{Silva, Jos\'e da}
\autorabr{SILVA, J.}

\cutter{S856m}
% Para gerar a ficha catalográfica sem o Código Cutter, basta 
% incluir uma % na linha acima e tirar a % da linha abaixo
%\cutter{ }

\local{S\~ao Carlos}
\data{2017}
% Quando for Orientador, basta incluir uma % antes do comando abaixo
\renewcommand{\orientadorname}{Orientadora:}
% Quando for Coorientadora, basta tirar a % utilizar o comando abaixo
%\newcommand{\coorientadorname}{Coorientador:}
\orientador{Profa. Dra. Elisa Gon\c{c}alves Rodrigues}
\orientadorcorpoficha{orientadora Elisa Gon\c{c}alves Rodrigues}
\orientadorficha{Rodrigues, Elisa Gon\c{c}alves, orient}
%Se houver co-orientador, inclua % antes das duas linhas (antes dos comandos \orientadorcorpoficha e \orientadorficha) 
%          e tire a % antes dos 3 comandos abaixo
%\coorientador{Prof. Dr. Jo\~ao Alves Serqueira}
%\orientadorcorpoficha{orientadora Elisa Gon\c{c}alves Rodrigues ;  co-orientador Jo\~ao Alves Serqueira}
%\orientadorficha{Rodrigues, Elisa Gon\c{c}alves, orient. II. Serqueira, Jo\~ao Alves, co-orient}

\notaautorizacao{AUTORIZO A REPRODU\c{C}\~AO E DIVULGA\c{C}\~AO TOTAL OU PARCIAL DESTE TRABALHO, POR QUALQUER MEIO CONVENCIONAL OU ELETR\^ONICO PARA FINS DE ESTUDO E PESQUISA, DESDE QUE CITADA A FONTE.}
\notabib{Ficha catalogr\'afica elaborada pela Biblioteca Prof. Achille Bassi, ICMC/USP, com os dados fornecidos pelo(a) autor(a)}

\newcommand{\programa}[1]{

% MPMp ==========================================================================
    \ifthenelse{\equal{#1}{MPMp}}{
     		\tipotrabalho{Disserta\c{c}\~ao (Mestrado em Ci\^encias)}
        \area{Matem\'atica}
				%\opcao{Nome da Opção}
        % O preambulo deve conter o tipo do trabalho, o objetivo, 
				% o nome da instituição, a área de concentração e opção quando houver
				\preambulo{Disserta\c{c}\~ao apresentada ao Instituto de Ci\^encias Matem\'aticas e de Computa\c{c}\~ao, Universidade de S\~ao Paulo - ICMC/USP, como parte dos requisitos para obten\c{c}\~ao do t\'itulo de Mestre em Ci\^encias - Programa de Mestrado Profissional em Matem\'atica.}
				\notaficha{Disserta\c{c}\~ao (Mestrado - Programa de Mestrado Profissional em Matem\'atica)}
    }{
% MPMe ==========================================================================
    \ifthenelse{\equal{#1}{MPMe}}{
     		\tipotrabalho{Dissertation (Master in Science)}
				\renewcommand{\areaname}{Concentration area:}
        \area{Mathematics}
				%\opcao{Nome da Opção}
        % O preambulo deve conter o tipo do trabalho, o objetivo, 
				% o nome da instituição, a área de concentração e opção quando houver
				\preambulo{Dissertation submitted to the Instituto de Ci\^encias Matem\'aticas e de Computa\c{c}\~ao, Universidade de S\~ao Paulo - ICMC/USP, in partial fulfillment of the requirements for the degree of the Master in Science - Mathematics Professional Master\'{}s Program.}
				\notaficha{Dissertation (Master - Mathematics Professional Master\'{}s Program)}
    }{
% DMAp ==========================================================================
    \ifthenelse{\equal{#1}{DMAp}}{
     		\tipotrabalho{Tese (Doutorado em Ci\^encias)}
        \area{Matem\'atica}
				%\opcao{Nome da Opção}
        % O preambulo deve conter o tipo do trabalho, o objetivo, 
				% o nome da instituição, a área de concentração e opção quando houver
				\preambulo{Tese apresentada ao Instituto de Ci\^encias Matem\'aticas e de Computa\c{c}\~ao, Universidade de S\~ao Paulo - ICMC/USP, como parte dos requisitos para obten\c{c}\~ao do t\'itulo de Doutor em Ci\^encias - Matem\'atica.}
				\notaficha{Tese (Doutorado - Programa de P\'os-Gradua\c{c}\~ao em Matem\'atica)}
    }{
% DMAe ==========================================================================
    \ifthenelse{\equal{#1}{DMAe}}{
     		\tipotrabalho{Thesis (Doctorate in Science)}
				\renewcommand{\areaname}{Concentration area:}
        \area{Mathematics}
				%\opcao{Nome da Opção}
        % O preambulo deve conter o tipo do trabalho, o objetivo, 
				% o nome da instituição, a área de concentração e opção quando houver
				\preambulo{Thesis submitted to the Instituto de Ci\^encias Matem\'aticas e de Computa\c{c}\~ao, Universidade de S\~ao Paulo - ICMC/USP, in partial fulfillment of the requirements for the degree of the Doctor in Science - Mathematics.}
				\notaficha{Thesis (Doctorate - Program in Mathematics)}
    }{
% MMAp ==========================================================================
    \ifthenelse{\equal{#1}{MMAp}}{
     		\tipotrabalho{Disserta\c{c}\~ao (Mestrado em Ci\^encias)}
        \area{Matem\'atica}
				%\opcao{Nome da Opção}
        % O preambulo deve conter o tipo do trabalho, o objetivo, 
				% o nome da instituição, a área de concentração e opção quando houver
				\preambulo{Disserta\c{c}\~ao apresentada ao Instituto de Ci\^encias Matem\'aticas e de Computa\c{c}\~ao, Universidade de S\~ao Paulo - ICMC/USP, como parte dos requisitos para obten\c{c}\~ao do t\'itulo de Mestre em Ci\^encias - Matem\'atica.}
				\notaficha{Disserta\c{c}\~ao (Mestrado - Programa de P\'os-Gradua\c{c}\~ao em Matem\'atica)}
    }{
% MMAe ==========================================================================
    \ifthenelse{\equal{#1}{MMAe}}{
     		\tipotrabalho{Dissertation (Master in Science)}
				\renewcommand{\areaname}{Concentration area:}
        \area{Mathematics}
				%\opcao{Nome da Opção}
        % O preambulo deve conter o tipo do trabalho, o objetivo, 
				% o nome da instituição, a área de concentração e opção quando houver
				\preambulo{Dissertation submitted to the Instituto de Ci\^encias Matem\'aticas e de Computa\c{c}\~ao, Universidade de S\~ao Paulo - ICMC/USP, in partial fulfillment of the requirements for the degree of the Master in Science - Mathematics.}
				\notaficha{Dissertation (Master - Program in Mathematics)}
    }{
% DESp ==========================================================================
    \ifthenelse{\equal{#1}{DESp}}{
     		\tipotrabalho{Tese (Doutorado em Estat\'istica)}
        \area{Estat\'istica}
				%\opcao{Nome da Opção}
        % O preambulo deve conter o tipo do trabalho, o objetivo, 
				% o nome da instituição, a área de concentração e opção quando houver
				\preambulo{Tese apresentada ao Instituto de Ci\^encias Matem\'aticas e de Computa\c{c}\~ao, Universidade de S\~ao Paulo - ICMC/USP e ao Departamento de Estat\'istica, Universidade Federal de S\~ao Carlos - DEs/UFSCar, como parte dos requisitos para obten\c{c}\~ao do t\'itulo de Doutor em Estat\'istica - Interinstitucional de P\'os-Gradua\c{c}\~ao em Estat\'istica.}
				\notaficha{Tese (Doutorado - Interinstitucional de P\'os-Gradua\c{c}\~ao em Estat\'istica)}
    }{
% DESe ==========================================================================
    \ifthenelse{\equal{#1}{DESe}}{
     		\tipotrabalho{Thesis (Doctorate in Statistics)}
				\renewcommand{\areaname}{Concentration area:}
        \area{Statistics}
				%\opcao{Nome da Opção}
        % O preambulo deve conter o tipo do trabalho, o objetivo, 
				% o nome da instituição, a área de concentração e opção quando houver
				\preambulo{Thesis submitted to the Instituto de Ci\^encias Matem\'aticas e de Computa\c{c}\~ao, Universidade de S\~ao Paulo - ICMC/USP and to the Departamento de Estat\'istica, Universidade Federal de S\~ao Carlos - DEs/UFSCar, in partial fulfillment of the requirements for the degree of the Doctor in Statistics - Joint Graduate Program in Statistics.}
				\notaficha{Thesis (Doctorate - Joint Graduate Program in Statistics)}
    }{     
% MESp ==========================================================================
    \ifthenelse{\equal{#1}{MESp}}{
     		\tipotrabalho{Disserta\c{c}\~ao (Mestrado em Estat\'istica)}
        \area{Estat\'istica}
				%\opcao{Nome da Opção}
        % O preambulo deve conter o tipo do trabalho, o objetivo, 
				% o nome da instituição, a área de concentração e opção quando houver
				\preambulo{Disserta\c{c}\~ao apresentada ao Instituto de Ci\^encias Matem\'aticas e de Computa\c{c}\~ao, Universidade de S\~ao Paulo - ICMC/USP e ao Departamento de Estat\'istica, Universidade Federal de S\~ao Carlos - DEs/UFSCar, como parte dos requisitos para obten\c{c}\~ao do t\'itulo de Mestre em Estat\'istica - Interinstitucional de P\'os-Gradua\c{c}\~ao em Estat\'istica.}
				\notaficha{Disserta\c{c}\~ao (Mestrado - Interinstitucional de P\'os-Gradua\c{c}\~ao em Estat\'istica)}
    }{
% MESe ==========================================================================
    \ifthenelse{\equal{#1}{MESe}}{
     		\tipotrabalho{Dissertation (Master in Statistics)}
				\renewcommand{\areaname}{Concentration area:}
        \area{Statistics}
				%\opcao{Nome da Opção}
        % O preambulo deve conter o tipo do trabalho, o objetivo, 
				% o nome da instituição, a área de concentração e opção quando houver
				\preambulo{Dissertation submitted to the Instituto de Ci\^encias Matem\'aticas e de Computa\c{c}\~ao, Universidade de S\~ao Paulo - ICMC/USP and to the Departamento de Estat\'istica- DEs, Universidade Federal de S\~ao Carlos - DEs/UFSCar, in partial fulfillment of the requirements for the degree of the Master in Statistics - Joint Graduate Program in Statistics.}
				\notaficha{Dissertation (Master - Joint Graduate Program in Statistics)}
    }{  
% DCCp ==========================================================================
    \ifthenelse{\equal{#1}{DCCp}}{
     		\tipotrabalho{Tese (Doutorado em Ci\^encias)}
        \area{Ci\^encias de Computa\c{c}\~ao e Matem\'atica Computacional}
				%\opcao{Nome da Opção}
        % O preambulo deve conter o tipo do trabalho, o objetivo, 
				% o nome da instituição, a área de concentração e opção quando houver
				\preambulo{Tese apresentada ao Instituto de Ci\^encias Matem\'aticas e de Computa\c{c}\~ao, Universidade de S\~ao Paulo - ICMC/USP, como parte dos requisitos para obten\c{c}\~ao do t\'itulo de Doutor em Ci\^encias - Ci\^encias de Computa\c{c}\~ao e Matem\'atica Computacional.}
				\notaficha{Tese (Doutorado - Programa de P\'os-Gradua\c{c}\~ao em Ci\^encias de Computa\c{c}\~ao e Matem\'atica Computacional)}				
    }{
% DCCe ==========================================================================
    \ifthenelse{\equal{#1}{DCCe}}{
     		\tipotrabalho{Thesis (Doctorate in Science)}
				\renewcommand{\areaname}{Concentration area:}
        \area{Computer Science and Computational Mathematics}
				%\opcao{Nome da Opção}
        % O preambulo deve conter o tipo do trabalho, o objetivo, 
				% o nome da instituição, a área de concentração e opção quando houver
				\preambulo{Thesis submitted to the Instituto de Ci\^encias Matem\'aticas e de Computa\c{c}\~ao, Universidade de S\~ao Paulo - ICMC/USP, in partial fulfillment of the requirements for the degree of the Doctor in Science - Program in Computer Science and Computational Mathematics.}
				\notaficha{Thesis (Doctorate - Program in Computer Science and Computational Mathematics)}
    }{			
% MCCp ==========================================================================
    \ifthenelse{\equal{#1}{MCCp}}{
     		\tipotrabalho{Disserta\c{c}\~ao (Mestrado em Ci\^encias)}
        \area{Ci\^encias de Computa\c{c}\~ao e Matem\'atica Computacional}
				%\opcao{Nome da Opção}
        % O preambulo deve conter o tipo do trabalho, o objetivo, 
				% o nome da instituição, a área de concentração e opção quando houver
				\preambulo{Disserta\c{c}\~ao apresentada ao Instituto de Ci\^encias Matem\'aticas e de Computa\c{c}\~ao, Universidade de S\~ao Paulo - ICMC/USP, como parte dos requisitos para obten\c{c}\~ao do t\'itulo de Mestre em Ci\^encias - Ci\^encias de Computa\c{c}\~ao e Matem\'atica Computacional.}
				\notaficha{Disserta\c{c}\~ao (Mestrado - Programa de P\'os-Gradua\c{c}\~ao em Ci\^encias de Computa\c{c}\~ao e Matem\'atica Computacional)}
    }{
% MCCe ==========================================================================
    \ifthenelse{\equal{#1}{MCCe}}{
     		\tipotrabalho{Dissertation (Master in Science)}
				\renewcommand{\areaname}{Concentration area:}
        \area{Computer Science and Computational Mathematics}
				%\opcao{Nome da Opção}
        % O preambulo deve conter o tipo do trabalho, o objetivo, 
				% o nome da instituição, a área de concentração e opção quando houver
				\preambulo{Dissertation submitted to the Instituto de Ci\^encias Matem\'aticas e de Computa\c{c}\~ao, Universidade de S\~ao Paulo - ICMC/USP, in partial fulfillment of the requirements for the degree of the Master in Science - Program in Computer Science and Computational Mathematics.}
				\notaficha{Dissertation (Master - Program in Computer Science and Computational Mathematics)}
    }{					
% Outros
		\tipotrabalho{Disserta\c{c}\~ao/Tese (Mestrado/Doutorado)}
		\area{Nome da \'Area}
		\opcao{Nome da Op\c{c}\~ao}
        % O preambulo deve conter o tipo do trabalho, o objetivo, 
				% o nome da instituição, a área de concentração e opção quando houver				
				\preambulo{Disserta\c{c}\~ao/Tese apresentada ao Instituto de Ci\^encias Matem\'aticas e de Computa\c{c}\~ao, Universidade de S\~ao Paulo - ICMC/USP, como parte dos requisitos para obten\c{c}\~ao do t\'itulo de Mestre/Doutor em Ci\^encias - Programa.}
				\notaficha{Disserta\c{c}\~ao/Tese (Mestrado/Doutorado - Programa)}
        }}}}}}}}}}}}}}}				







