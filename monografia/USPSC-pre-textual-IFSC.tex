%% USPSC-pre-textual-IFSC.tex
%% Camandos para definição do tipo de documento (tese ou dissertação), área de concentração, opção, preâmbulo, titulação 
%% referentes ao Programa de Pós-Graduação o IFSC
\instituicao{Instituto de F\'isica de S\~ao Carlos, Universidade de S\~ao Paulo}
\unidade{INSTITUTO DE F\'ISICA DE S\~AO CARLOS}
\unidademin{Instituto de F\'isica de S\~ao Carlos}
\universidademin{Universidade de S\~ao Paulo}
\notafolharosto{Vers\~ao original}
%Para versão original em inglês, comente do comando/declaração 
%     acima(inclua % antes do comando acima) e tire a % do 
%     comando/declaração abaixo no idioma do texto
%\notafolharosto{Original version} 
%Para versão corrigida, comente do comando/declaração da 
%     versão original acima (inclua % antes do comando acima) 
%     e tire a % do comando/declaração de um dos comandos 
%     abaixo em conformidade com o idioma do texto
%\notafolharosto{Vers\~ao corrigida \\(Vers\~ao original dispon\'ivel na Unidade que aloja o Programa)}
%\notafolharosto{Corrected version \\(Original version available on the Program Unit)}

% ---
% dados complementares para CAPA e FOLHA DE ROSTO
% ---
\universidade{UNIVERSIDADE DE S\~AO PAULO}
\titulo{Modelo para teses e disserta\c{c}\~oes em \LaTeX\ utilizando a classe USPSC para o IFSC}
\titleabstract{Model for theses and dissertations in \LaTeX\ using the USPSC class to the IFSC}
\autor{Jos\'e da Silva}
\autorficha{Silva, Jos\'e da}
\autorabr{SILVA, J.}

\cutter{S856m}
% Para gerar a ficha catalográfica sem o Código Cutter, basta 
% incluir uma % na linha acima e tirar a % da linha abaixo
%\cutter{ }

\local{S\~ao Carlos}
\data{2017}
% Quando for Orientador, basta incluir uma % antes do comando abaixo
\renewcommand{\orientadorname}{Orientadora:}
% Quando for Coorientadora, basta tirar a % utilizar o comando abaixo
%\newcommand{\coorientadorname}{Coorientador:}
\orientador{Profa. Dra. Elisa Gon\c{c}alves Rodrigues}
\orientadorcorpoficha{orientadora Elisa Gon\c{c}alves Rodrigues}
\orientadorficha{Rodrigues, Elisa Gon\c{c}alves, orient}
%Se houver co-orientador, inclua % antes das duas linhas (antes dos comandos \orientadorcorpoficha e \orientadorficha) 
%          e tire a % antes dos 3 comandos abaixo
%\coorientador{Prof. Dr. Jo\~ao Alves Serqueira}
%\orientadorcorpoficha{orientadora Elisa Gon\c{c}alves Rodrigues ;  co-orientador Jo\~ao Alves Serqueira}
%\orientadorficha{Rodrigues, Elisa Gon\c{c}alves, orient. II. Serqueira, Jo\~ao Alves, co-orient}

\notaautorizacao{AUTORIZO A REPRODU\c{C}\~AO E DIVULGA\c{C}\~AO TOTAL OU PARCIAL DESTE TRABALHO, POR QUALQUER MEIO CONVENCIONAL OU ELETR\^ONICO PARA FINS DE ESTUDO E PESQUISA, DESDE QUE CITADA A FONTE.}
\notabib{Ficha catalogr\'afica revisada pelo Servi\c{c}o de Biblioteca e Informa\c{c}\~ao Prof. Bernhard Gross, com os dados fornecidos pelo(a) autor(a)}

\newcommand{\programa}[1]{

% DFA ==========================================================================
    \ifthenelse{\equal{#1}{DFA}}{
     	\tipotrabalho{Tese (Doutorado em Ci\^encias)}
        \area{F\'isica Aplicada}
		%\opcao{Nome da Opção}
        % O preambulo deve conter o tipo do trabalho, o objetivo, 
		% o nome da instituição, a área de concentração e opção quando houver
		\preambulo{Tese apresentada ao Programa de P\'os-Gradua\c{c}\~ao em F\'isica do Instituto de F\'isica de S\~ao Carlos da Universidade de S\~ao Paulo, para obten\c{c}\~ao do t\'itulo de Doutor em Ci\^encias.}
		\notaficha{Tese (Doutorado - Programa de P\'os-Gradua\c{c}\~ao em F\'isica Aplicada)}
    }{
% MFA ===========================================================================
        \ifthenelse{\equal{#1}{MFA}}{
	        \tipotrabalho{Disserta\c{c}\~ao (Mestrado em Ci\^encias)}
	        \area{F\'isica Aplicada}
			%\opcao{Nome da Opção}
	        % O preambulo deve conter o tipo do trabalho, o objetivo, 
			% o nome da instituição, a área de concentração e opção quando houver
			\preambulo{Disserta\c{c}\~ao apresentada ao Programa de P\'os-Gradua\c{c}\~ao em F\'isica do Instituto de F\'isica de S\~ao Carlos da Universidade de S\~ao Paulo, para obten\c{c}\~ao do t\'itulo de Mestre em Ci\^encias.}
			\notaficha{Disserta\c{c}\~ao (Mestrado - Programa de P\'os-Gradua\c{c}\~ao em F\'isica Aplicada)}
        }{
% DFAFC ==========================================================================
    \ifthenelse{\equal{#1}{DFAFC}}{
     	\tipotrabalho{Tese (Doutorado em Ci\^encias)}
        \area{F\'isica Aplicada}
        \opcao{F\'isica Computacional}
        % O preambulo deve conter o tipo do trabalho, o objetivo, 
		% o nome da instituição, a área de concentração e opção quando houver
		\preambulo{Tese apresentada ao Programa de P\'os-Gradua\c{c}\~ao em F\'isica do Instituto de F\'isica de S\~ao Carlos da Universidade de S\~ao Paulo, para obten\c{c}\~ao do t\'itulo de Doutor em Ci\^encias.}
		\notaficha{Tese (Doutorado - Programa de P\'os-Gradua\c{c}\~ao em F\'isica Aplicada)}
    }{
% MFAFC ===========================================================================
        \ifthenelse{\equal{#1}{MFAFC}}{
			\tipotrabalho{Disserta\c{c}\~ao (Mestrado em Ci\^encias)}
	        \area{F\'isica Aplicada}
	        \opcao{F\'isica Computacional}
	        % O preambulo deve conter o tipo do trabalho, o objetivo, 
			% o nome da instituição, a área de concentração e opção quando houver
			\preambulo{Disserta\c{c}\~ao apresentada ao Programa de P\'os-Gradua\c{c}\~ao em F\'isica do Instituto de F\'isica de S\~ao Carlos da Universidade de S\~ao Paulo, para obten\c{c}\~ao do t\'itulo de Mestre em Ci\^encias.}
			\notaficha{Disserta\c{c}\~ao (Mestrado - Programa de P\'os-Gradua\c{c}\~ao em F\'isica Aplicada)}
        }{
% DFAFBp ===========================================================================
        \ifthenelse{\equal{#1}{DFAFBp}}{
			\tipotrabalho{Tese (Doutorado em Ci\^encias)}
	        \area{F\'isica Aplicada}
	        \opcao{F\'isica Biomolecular}
	        % O preambulo deve conter o tipo do trabalho, o objetivo, 
			% o nome da instituição, a área de concentração e opção quando houver
			\preambulo{Tese apresentada ao Programa de P\'os-Gradua\c{c}\~ao em F\'isica do Instituto de F\'isica de S\~ao Carlos da Universidade de S\~ao Paulo, para obten\c{c}\~ao do t\'itulo de Doutor em Ci\^encias.}
			\notaficha{Tese (Doutorado - Programa de P\'os-Gradua\c{c}\~ao em F\'isica Aplicada)}
        }{				
% DFAFBe ===========================================================================
        \ifthenelse{\equal{#1}{DFAFBe}}{
			\renewcommand{\areaname}{Concentration area:}
			\renewcommand{\opcaoname}{Option:}
			\tipotrabalho{Thesis (Doctor in Science)}
	        \area{Applied Physics}
	        \opcao{Biomolecular Physics}
	        % O preambulo deve conter o tipo do trabalho, o objetivo, 
			% o nome da instituição, a área de concentração e opção quando houver
			\preambulo{Thesis presented to the Graduate Program in Physics at the Instituto de F\'isica de S\~ao Carlos, Universidade de S\~ao Paulo, to obtain the degree of Doctor in Science.}
			\notaficha{Thesis (Doctorate - Graduate Program in Applied Physics)}
        }{				
% MFAFB ===========================================================================
        \ifthenelse{\equal{#1}{MFAFB}}{
	        \tipotrabalho{Disserta\c{c}\~ao (Mestrado em Ci\^encias)}
	        \area{F\'isica Aplicada}
	        \opcao{F\'isica Biomolecular}
	        % O preambulo deve conter o tipo do trabalho, o objetivo, 
			% o nome da instituição, a área de concentração e opção quando houver
			\preambulo{Disserta\c{c}\~ao apresentada ao Programa de P\'os-Gradua\c{c}\~ao em F\'isica do Instituto de F\'isica de S\~ao Carlos da Universidade de S\~ao Paulo, para obten\c{c}\~ao do t\'itulo de Mestre em Ci\^encias.}
			\notaficha{Disserta\c{c}\~ao (Mestrado - Programa de P\'os-Gradua\c{c}\~ao em F\'isica Aplicada)}
        }{
				
% DFB ==========================================================================
    \ifthenelse{\equal{#1}{DFB}}{
     	\tipotrabalho{Tese (Doutorado em Ci\^encias)}
        \area{F\'isica B\'asica}
		%\opcao{Nome da Opção}
        % O preambulo deve conter o tipo do trabalho, o objetivo, 
		% o nome da instituição, a área de concentração e opção quando houver				
		\preambulo{Tese apresentada ao Programa de P\'os-Gradua\c{c}\~ao em F\'isica do Instituto de F\'isica de S\~ao Carlos da Universidade de S\~ao Paulo, para obten\c{c}\~ao do t\'itulo de Doutor em Ci\^encias.}
		\notaficha{Tese (Doutorado - Programa de P\'os-Gradua\c{c}\~ao em F\'isica B\'asica)}
    }{
% MFB ===========================================================================
        \ifthenelse{\equal{#1}{MFB}}{
	        \tipotrabalho{Disserta\c{c}\~ao (Mestrado em Ci\^encias)}
	        \area{F\'isica B\'asica}
			%\opcao{Nome da Opção}
	        % O preambulo deve conter o tipo do trabalho, o objetivo, 
			% o nome da instituição, a área de concentração e opção quando houver				
			\preambulo{Disserta\c{c}\~ao apresentada ao Programa de P\'os-Gradua\c{c}\~ao em F\'isica do Instituto de F\'isica de S\~ao Carlos da Universidade de S\~ao Paulo, para obten\c{c}\~ao do t\'itulo de Mestre em Ci\^encias.}
			\notaficha{Disserta\c{c}\~ao (Mestrado - Programa de P\'os-Gradua\c{c}\~ao em F\'isica B\'asica)}
        }{                
% Outros
				\tipotrabalho{Disserta\c{c}\~ao/Tese (Mestrado/Doutorado)}
				\area{Nome da \'Area}
				\opcao{Nome da Op\c{c}\~ao}
		        % O preambulo deve conter o tipo do trabalho, o objetivo, 
				% o nome da instituição, a área de concentração e opção quando houver
				\preambulo{Disserta\c{c}\~ao/Tese apresentada ao Programa de P\’{\o}s-Gradua\c{c}\~ao em F\’{\i}sica do Instituto de F\’{\i}sica de S\~ao Carlos da Universidade de S\~ao Paulo, para obtenç\c{c}\~ao do t\’{\i}tulo de Mestre/Doutor em Ci\^encias.}
				\notaficha{Disserta\c{c}\~ao/Tese (Mestrado/Doutorado - Programa de P\'os-Gradua\c{c}\~ao em Nome da \'Area)}
        }}}}}}}}}}
				
				






